\chapter{Реализация}
\label{chap:implementation}
\section{Интеграция}
Полученная модель состоит из следующих составных частей:
\begin{itemize}
	\item спинной мозг
	\item продолговатый мозг
	\item Варолиев мост
	\item нейромышечные соединения
	\item чувствительные нейроны периферической нервной системы, восходящие по спинному мозгу, через таламокортикальные пути к моторной и сенсорной областям коры мозга
\end{itemize}

В листинге ниже представлен код, представляющий эти данные в удобном для проекта \en{NEUCOGAR} виде: 
\begin{lstlisting}
spine = ({k_name: 'Spinal cord [Glu1]'},    # to lower motor neurons
	{k_name: 'Spinal cord [Glu2]'},    # upwards to the thalamus
	{k_name: 'Spinal cord [GABA]'},)   # pain reduction, internal use only
spine_Glu1, spine_Glu2, spine_GABA = range(3)

nmj = ({k_name: 'Muscle [Glu]'},)
nmj_Glu = 0

cellBodies = ({k_name: 'Muscle [Glu]'},)
cellBodies_Glu = 0

medulla = ({k_name: 'Medulla [GABA]'}, )
medulla_GABA = 0

reticular_formation = ({k_name: 'Reticular formation [Glu]'},
	{k_name: 'Reticular formation [5HT]'},)
reticular_formation_Glu, reticular_formation_5HT = range(2)

pons = ({k_name: 'Pons [Glu]'},
	{k_name: 'Pons [5HT]'})
pons_Glu, pons_5HT = range(2)


pons[pons_5HT][k_NN] = 1000
pons[pons_Glu][k_NN] = 1500
medulla[medulla_GABA][k_NN] = 4000

spine_NN = 15000
spine[spine_Glu1][k_NN] = spine_NN / 2
spine[spine_Glu2][k_NN] = spine_NN / 3
spine[spine_GABA][k_NN] = spine_NN / 6

nmj[nmj_Glu][k_NN] = spine_NN
cellBodies[cellBodies_Glu][k_NN] = spine[spine_Glu1][k_NN]
reticular_formation[reticular_formation_Glu][k_NN] = 1000
reticular_formation[reticular_formation_5HT][k_NN] = 1000
\end{lstlisting}
Все нужные пути и отделы мозга представлены в виде словарей, где парами ключ-значение являются количество нейронов в отделе и название, \en{k\_NN} и \en{k\_name} соответственно. 
Представленные данные о количестве нейронов приведены в соответствие с общей тенденцией проекта к численности нейронов.

Остается соединить данные системы с готовой кодовой базой проекта, однако, стоит обратить внимание на Варолиев мост, он уже представлен в коде, поэтому он был расширен. В частности, к уже существующим cеротониновым рецепторам добавлена необходимая для нисходящего пути часть.

В следующем листинге представлен код, создающий нужные для работы афферентных и эфферентных путей нейронные связи между частями мозга:
\begin{lstlisting}
# * * * ЭФФЕРЕНТНЫЕ * * *
# * * * КОРТИКОСПИНАЛЬНЫЙ ТРАКТ * * *
# connect(motor[motor_Glu1], medulla[medulla_GABA], syn_type=GABA, weight_coef=0.01)
connect(motor[motor_Glu1], spine[spine_Glu1], syn_type=Glu)
connect(spine[spine_Glu1], nmj[nmj_Glu], syn_type=Glu)

# # * * * КОРТИКОБУЛЬБАРНЫЙ ТРАКТ * * *
connect(motor[motor_Glu0], medulla[medulla_GABA], syn_type=Glu)
# # * * * РЕТИКУЛОСПИНАЛЬНЫЙ ТРАКТ * * *
connect(pons[pons_Glu], spine[spine_GABA], syn_type=Glu)
connect(medulla[medulla_GABA], spine[spine_GABA], syn_type=GABA)

# * * * АФФЕРЕНТНЫЕ * * *
# * * * СПИНОТАЛАМИЧЕСКИЙ ПУТЬ * * *
connect(cellBodies[cellBodies_Glu], spine[spine_Glu2], syn_type=Glu)
connect(spine[spine_Glu2], thalamus[thalamus_Glu], syn_type=Glu)
\end{lstlisting}

По представленному коду видны проекции нейронных путей из моторной коры к спинному мозгу, и, в конечном итоге, к мышцам. 
Также восходящий путь от клеток тела к таламусу, для интеграции сигнала, и дальнейшей отправки к коре головного мозга, в основном, к его сенсорной части, которая реализована, к сожалению, в ограниченном виде.

\section{Тесты}
Было проведено несколько серий тестов: 
\begin{itemize}
	\item серия из нескольких запусков с различным общим количеством нейронов для проверки быстродействия и потребления памяти
	\item серия экспериментов для проверки полученной модели, описанные в \ref{section:validation}
\end{itemize}
