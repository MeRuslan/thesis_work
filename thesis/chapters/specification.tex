\chapter{Спецификация}
\label{chap:specification}
\section{Общая характеристика симуляции}
Запускается программа локально, для параллельных вычислений используется \en{openMP}.
Симуляция проводится при следующих характеристиках компьютера и симуляции: \ref{tab:conditions}

\begin{table}
	\caption{Условия запуска}
	\label{tab:conditions}
	\begin{tabular}{l c r}
		\toprule
		Параметр & Значение \\
		\midrule
		Начало & 0 мс. \\
		Конец & 1000 мс. \\
		Дискретизация по времени & 10 мс. \\
		Объем оперативной памяти & 5 Гбайт \\
		Процессор & AMD A10-4655M \\
		Количество ядер & 4 \\
		\bottomrule
	\end{tabular}
\end{table}

\section{Основные функции}
Самой частой для данного сценария использования командой является: 
\begin{lstlisting}
	Connect(pre, post, conn_spec=None, syn_spec=None, model=None)
\end{lstlisting}
Значения параметров описаны в таблице \ref{tab:connect_params}. 
Команда соединяет две группы нейронов согласно заданным правилам.

Стоит отметить также функцию соединения детектора с нейронами сети для замера полученных результатов симуляции.
\begin{lstlisting}
def connect_detector(part):
	name = parts_dict[part]
	number = len(part) if len(part) < N_rec else N_rec
	spikedetectors[name] = nest.Create('spike_detector', params=detector_param)
	nest.Connect(part[:number], spikedetectors[name])
\end{lstlisting}
Функция соединяет к заданному в словаре part набору нейронов детектор, который считывает показания с \en{number} нейронов.

Список ключей в словаре \en{part} и значения каждого ключа заданы в таблице \ref{tab:part_params}

\begin{table}
	\caption{Параметры соединения нейронов}
	\label{tab:connect_params}
	\begin{tabular}{l c r}
		\toprule
		Параметр & Значение \\
		\midrule
		pre & Пресинаптические нейроны \\
		post & Постсинаптические нейроны \\
		conn\_spec & Словарь, определяющий правила подключения \\
		syn\_spec & Словарь, описывающий синапс \\
		model & Тип синапса \\
		\bottomrule
	\end{tabular}
\end{table}

\begin{table}
	\caption{Параметры групп}
	\label{tab:part_params}
	\begin{tabular}{l c r}
		\toprule
		Параметр & Значение \\
		\midrule
		k\_name & Название \\
		k\_NN & Число нейронов \\
		\bottomrule
	\end{tabular}
\end{table}

\section{Пути моделирования}
Основные отделы мозга, имеющие влияние на двигательную активность были ранее описаны в \ref{section:motor_things}.
В данном случае необходимо интегрировать группу нейронов, относящихся к: спинному мозгу, стволу головного мозга, моторной коре, нейромускулярным соединениям и чувствительным клеткам периферической нервной системы с основной кодовой базой проекта.

\subsection{Кортикоспинальный тракт}
В коде NEUCOGAR имеется реализации моторной коры. именно отсюда берет начало данный путь, известно, что он отвечает за исполнение добровольных движений, тип соединений --- возбуждающий Glu синапс.

\subsection{Афферентный кортикальный вход}
Данный путь ответственен за отдачу сенсорной информации в кору головного мозга через таламокортикальное пути в двигательные и сенсорные области, так как таламокортикальное пути в проекте реализованы, необходимо задать соединения данного пути с необходимым участком таламуса.

\section{Используемые типы синапсов}
В проекте используется несколько типов синапсов, однако, так как все, что касается дофамина, серотонина и норадреналина уже реализовано, для реализации данного проекта необходимо лишь два:
\begin{itemize}
	\item GABA, нейротрансмиттер --- гамма-аминомасляная кислота (ГАМК), тормозной
	\item Glu, возбуждающий нейротрансмиттер --- глутаминовая кислота
\end{itemize}

