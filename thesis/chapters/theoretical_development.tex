\chapter{Теоретическое обоснование}
\label{chap:theoretical_development}
\section{NEST}
\en{NEST} является симулятором нейросетевых моделей, он фокусируется в большей степени на динамике и структуре нейронных систем, чем на точной морфологии отдельных нейронов. 
Основное преимущество данного фреймворка в том, что в \en{NEST} реализованы практически все необходимые для проведения эксперимента типы нейронов, синапсов, нейромодуляторов и нейромедиаторов. 

Кроме того, \en{NEST} имеет удобный \en{API} для языка \en{Python}, он называется \en{PyNEST}, и предоставляет набор команд интерпретатору \en{Python}, которые дают доступ к описанию эксперимента, и контролю над его проведением. 
Сама симуляция при этом запускается в оптимизированном ядре моделирования \en{NEST}.
Моделирование NEST пытается следовать логике эксперимента в действительности, который просчитывается компьютером. Однако при этом исследуемая нейронная система определяется экспериментатором.

Он подходит для моделирования:
\begin{itemize}
	\item обработки информации, например в коре мозга
	\item моделей динамики активности в нейронной произвольной сети
	\item обучения и пластичности
\end{itemize}

\section{NEUCOGAR, интеграция}
В коде когнитивной архитектуры \en{NEUCOGAR} реализованы пути распространения основных нейромодуляторов, благодаря этому при интеграции данной модели моторной активности с кодовой базой \en{NEUCOGAR} можно исследовать взаимодействие моторной и лимбической систем мозга. 
Для обоснования работоспособности данной модели нужно исследовать реакцию системы на различные ситуации. 


\section{Валидация}
\label{section:validation}
Будет проведено несколько экспериментов для мониторинга реакции агента на различные ситуации.

\subsection{Проверка влияния афферентных путей}\label{section:validation2}
\begin{itemize}
	 \item ввод системы в состояние \enquote{удивления}
	 \item внезапный и высокочастотный стимул со стороны афферетного пути, сигнализирующего о внезапном интенсивном взаимодействии со средой
\end{itemize}
Ожидается активизация таламуса, с последующим воздействием на остальную систему эмоциональной оценки.

\subsection{Проверка взаимосвязи между моторным выводом и оценкой}
\label{section:validation1}
\begin{itemize}
	\item ввод системы в состояние \enquote{удивления}
	\item посмотреть на активацию потенциалов действия в спинном мозге и нейромышечных соединениях
	\item задать более жесткий стимул для активации мышц в виде подключения генераторов к моторной коре мозга
	\item рассмотреть результаты
\end{itemize}
С применением прямого стимула совмещенного с эмоциональной оценкой \enquote{удивления} результирующая частота активации потенциалов действия в нейронах усилится, превзойдет результаты стимуляции по отдельности и эмоциональной оценки, и прямого воздействия на двигательную кору.