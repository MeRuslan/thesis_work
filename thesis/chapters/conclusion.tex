\cleardoublepage
\phantomsection
\addcontentsline{toc}{chapter}{Заключение}
\chapter*{Заключение}
\label{chap:conclusion}
В ходе постановки экспериментов в рамках данной работы была разработана симуляция ответственных за моторную подсистему областей мозга, и запущена вместе с системой эмоциональной оценки в контексте проекта когнитивной архитектуры \en{NEUCOGAR}.

\section{Рекомендации к дальнейшему исследованию}
Постановка экспериментов в малых масштабах несет с собой ограничение применимости суждений о работе данного подхода в целом при значительно больших масштабах. Полученный в результате код был запущен со следующими параметрами:
\begin{itemize}
	\item 1000-10000 нейронов
	\item 30000-2000000 синапсов
\end{itemize}
Для дальнейшего исследования данной области необходимо масштабировать симуляцию на биологически реалистичное количество нейронов в мозге для достижения относительно точной репликации процессов в биологическом мозге в цифровой среде.

Также, для практической применимости полученных результатов к прикладной области информатики необходимо наличие в архитектуре системы восприятия состояния среды, с возможностью обучения агента выдаче \enquote{эмоциональной} реакции на определенные стимулы.

\section{Перспективы применения}
Полученная в результате программа оснащенного биологически инспирированной когнитивной архитектурой агента может быть использована для исследования:
\begin{itemize}
	\item его поведения в целом
	\item поведения определенного рода среде 
	\item механизмов адаптации к среде
	\item реакции на стимулы, вызывающие различные \enquote{эмоциональные} отклики
	\item для валидации	данного подхода к проектированию интеллектуальных систем
\end{itemize}

\section{Вывод}
Разработка данного проекта позволила оценить необходимое для оперирования когнитивной архитектуры ресурсы вычислительной системы, включая затраченное процессорное время и используемую оперативную память.

Кроме этого, она также позволяет взглянуть на функционирование основных путей и функциональных блоков организации двигательной активности в соединении с функциональными элементами лимбической системы, включая эмоциональную составляющую.
Влияние эмоциональных стимулов на обработку моторного вывода и ответной реакции на раздражитель.

